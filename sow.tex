\documentclass[12pt]{article}

\begin{document}

Don Blair: Application for R
Application for Research Affliate Position \\
Center for Civic Media \\
\section*{Statement of Work}

As a Research Affliate with the Center for Civic Media, my aim would be to deepen a set of collaborative projects that have already begun to coalesce within the welcoming, open environment maintained at the Center. As an Affliate, I would hope to contribute to an array of emerging and existing projects and themes (detailed below) in several modes: through the writing of blog posts, working papers, and book chapters; by assisting others in their projects (through technical development and strategizing); and by engaging in and facilitating relevant dialogues within the Civic Community.  An initial project list - surely to undergo revision over time - might look like the following:

\paragraph{Support structures for community science.} The aim here is to build on initiatives already underway within the Civic community by connecting them to complementary intiatives in the Public Lab and FarmHack communities.  Ongoing and planned Civic initiatives focused on civic monitoring projects in the US, Brasil, Kenya, China and elsewhere might usefully draw upon the technical and social support structures for community science being developed within Public Lab's Open Water, Open Air, and Open Land initiatives; indeed, there is already work being done on the Open Water project at the Center.  FarmHack's Growing Clean Water adds a long-term remediation perspective, and the community around food production, to this collaborative space.  My aim would be to leverage the expertise and talent in Civic to convene collaborations around the design processes, community education practices, and various notions of accessibility that undergird these projects.  I look forward to continuing towork with Heather on this project, and see strong overlaps with the work being done by Emilie, Jude, Erhardt, and Nathon in this area.

\paragraph{Hierarchies, labor, and the role of institutions in the knowledge commons.} As communities gain access to more powerful, decentralized tools for monitoring, manufacturing, energy production, and remediation, and as existing institutions seek to accomodate these new sources of data and material output, issues of regulation, certification, calibration, data veracity, and safety are coming to the fore.  I would like to use a set of ongoing, concrete projects, in the areas of 

\begin{itemize}
\item environmental monitoring (Public Lab);
\item remediation, sanitation, and food production (Farm Hack); and 
\item human health (Innovations in Mother Child Health) 
\end{itemize}

as anchors for the development of frameworks and analyses (via essays and working papers) and ongoing community dialogues (via meet-ups and colloquia).  For these projects, the gathered expertise and infrastructure at the Center would be a great boon; working with Erhardt, Nathan, Catherine, Tal, and Willow, Alexis. 

\paragraph{Civic Engagement Through Narrative.} In my own recent work, I have begun to see (after being exposed to the vision of people in the Civic Community) strong and useful connections between the aspirations of community science communities like Public Lab, and the tools, methods, and approach already being employed within professional and academic fields whose focus is on narrative, rhetoric, interpretation, and communicating with specific audiences:  the fields of journalism, communications, literature, design, art, and much of the humanities.  As part of the scope of a planned Environmental Storytelling Institute (NSF or other funding pending), I would like to include research at Civic into the incorporation of these narrative techniques and insights into the civic engagement space. Catherine's work, the networked stories group, the Future of News Initiative, Matt, Adrienne, Rahul. 

I would also hope to prompt dialogue related to the above themes through blog posts (on the Civic blog) as well as short essays (on, for example, medium.com), written in collaboration with people from the Civic community, on topics like:

\begin{description}

\item[``Hidden Needs.''] Lessons from the 
\item[``Dangerous Toys.'']
\item[``Social Network Regulation?''] 

\end{description}




\end{document}

